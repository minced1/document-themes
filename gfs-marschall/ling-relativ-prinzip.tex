% To build everything run
%	latexmk -xelatex
% To clean temporary files run
%	latexmk -c
\documentclass[12pt]{scrreprt}

% German locale *************************
\usepackage[T1]{fontenc}
\usepackage[ngerman]{babel}
%****************************************

% Use Verbose citation ******************
%\usepackage[natbibapa]{apacite}
\usepackage{csquotes}
\usepackage{url}
\usepackage[
	backend		= biber,
	style			= verbose,
	autocite	= footnote
	]{biblatex}
	\bibliography{ling-relativ-prinzip}
%****************************************

% Add images ************
\usepackage{graphicx}
%****************************************

% Configure captions ********************
\usepackage[
	format			= hang,
	labelfont		= bf,
	font				= bf,
	figurename	= Abb.,
	tablename		= Tab.
	]{caption}
%****************************************

% Blind text*****************************
\usepackage{lipsum}
%****************************************

% Define itemize bullets ****************
\renewcommand{\labelitemi}{$\bullet$}
\renewcommand{\labelitemii}{$\circ$}
\renewcommand{\labelitemiii}{$\bullet$}
\renewcommand{\labelitemiv}{$\circ$}
%****************************************

% Define enumerate numbers ****************
\renewcommand{\labelenumii}{\arabic{enumi}.\arabic{enumii}}
\renewcommand{\labelenumiii}{\arabic{enumi}.\arabic{enumii}.\arabic{enumiii}.}
\renewcommand{\labelenumiv}{\arabic{enumi}.\arabic{enumii}.\arabic{enumiii}.\arabic{enumiv}.}
%****************************************

% Configure fonts ***********************
% Comment out when fonts are not
% found or use different fonts
%\usepackage{lmodern}
\usepackage{fontspec}
	\setmainfont{TeX Gyre Pagella}
	\setsansfont{TeX Gyre Heros}
%****************************************

% Define layout *************************
\usepackage{setspace}
	\setstretch{1.41}
\usepackage[
	a4paper,
	left		= 2.5cm,
	right		= 3cm,
	top			= 2cm,
	bottom	= 2cm,
	%includeheadfoot,
	%showframe
	]{geometry}
%****************************************

%****************************************
\usepackage{scrlayer-scrpage}
\usepackage[bottom]{footmisc}
	\clearpairofpagestyles

	\setkomafont{pageheadfoot}{\rmfamily}
	%\setkomafont{pagehead}{\bfseries}
	\setkomafont{pagination}{}

	\KOMAoptions{
	   headsepline	= true,
	   footsepline	= false,
	   %plainfootsepline	= true,
	}

	\automark[chapter]{chapter}

	\ihead{\headmark}
	\ohead{\pagemark}
%****************************************

% Make titlepage ************************
\usepackage{titlepage}
	\ititle{Das linguistische Relativitätsprinzip}
	\isubtitle{Sapir-Whorf-Hypothese} % Optional
	\iauthor{Jasper Gude}
	\idate{\today}
	\irefnr{}
	\iaddress{Hockenheim} % Only for ireports!
%****************************************

\begin{document}

\makeititle
%***************
\begin{center}
	\sffamily\bfseries{Eigenständigkeitserklärung}
\end{center}
Ich versichere, dass ich diese Ausarbeitung selbständig verfasst, alle aus
anderen Werken wörtlich oder sinngemäß entnommenen Stellen unter Angabe der
Quelle als Entlehnung kenntlich gemacht und andere als die angegebenen
Hilfsmittel nicht benutzt habe.

Oftersheim, \today, Unterschrift
%***************
\tableofcontents
%***************
\listoffigures
%***************
%\listoftables
%***************
\chapter{Vorwort}
	\label{chap:vorwort}
\blockquote{
\enquote{Wie hängen Sprache, Denken und Wirklichkeit
zusammen?} – Das \enquote{linguistische Relativitätsprinzip} von Benjamin Le
Whorf
\medskip\newline
Inwiefern ist Sprache ein Medium der Erkenntnis?
Was ist dahingehend die Ansicht der aktuellen Neurowissenschaft?}
Als GFS-Thema hat mich diese Fragestellung sehr angeprochen.
Die Deutschthemen der Klassenstufen davor erschienen mir oft entweder sehr
trocken, oder sehr willkürlich.

Bei diesem Thema gefiel mir direkt der wissenschaftliche Ansatz, der in die
psychologische und neurologische Richtung geht, auch wenn ich zugeben muss, dass
mich diese Bereiche der Wissenschaft zwar interessieren, ich mich damit aber nie
besonders tiefreichend beschäftigt habe.

Ein nettes Nebenprodukt dieser GFS wird also die Erweiterung meines
wissenschaftlichen Horizonts sein.
\chapter{Benjamin Lee Whorf}
\label{chap:bjwhorf}
Benjamin Lee Whorf wird am 24. April 1897 in Winthrop, Massachussetts geboren
und starb früh im Alter von 44 Jahren in Wethersfield, Connecticut.
\begin{figure}[!htb]
	\centering
	\includegraphics[width=0.25\textwidth]{whorf}
	\caption[Benjamin Lee Whorf {\autocite{image:B_L_Whorf}}]{Benjamin Lee Whorf\footnotemark}
	\label{fig:whorf}
\end{figure}
\footcitetext{image:B_L_Whorf}\\
Zu seinen populärsten Arbeiten als Amateurlinguist zählen vor allem die Arbeiten,
zu den amerikanischen Eingeborenensprachen, wie etwa die der Hopi, als auch die
These der \enquote{sprachlichen Relativität}, um die es in dieser Facharbeit
gehen soll.

	\section{Leben und Werk}
	\label{sec:lebenuwerk}
	Whorf schließt 1928 das Chemietechnik-Studium am \textit{Massachusetts Institute of
	Technology (MIT)} ab und fängt bei einer Versicherungsgesellschaft als
	Brandverhütungsinspektor an, zu arbeiten. Bei diesem Unternehmen wird er
	zeitlebens bleiben und Karriere machen.

	Sein Interesse an der Wissenschaft lindert dies jedoch nicht. Whorf lernt
	Hebräisch und forscht im Bereich der aztekischen Nahua-Sprachen und der
	Maya-Sprachen. Seine Erkenntnisse kommen jedoch zu früh und treffen auf wenig
	Begeisterung. Erst nach seinem Tod entwickelt seine Arbeit Popularität und
	wird aktiv diskutiert und kritisiert.
		\subsection{Einflüsse}
		\label{sec:einfluesse}
		Zu Benjamin Whorfs größten Einflüssen zählt Edward Sapir. Als dieser zum
		\textit{Sterling Professor für Linguistik und Anthropologie} an der
		Universität Yale ernannt wird, fängt Whorf sofort an, bei ihm amerikanische
		indianische Linguistik zu studieren.

		Dabei macht Sapir ihn auf die Sprache der Hopi aufmerksam, die er durch einen
		Informanten in New York lernt. In der Zeit danach entstehen mehrere Arbeiten,
		die zum Teil erst nach seinem Tod veröffentlicht werden.

		Den Denkanstoß in Richtung der Entwicklung seiner Konzepte, die später als
		Sapir-Whorf-Hypothese bekannt wird, gibt ihm seine Arbeit für die
		Versicherungsgesellschaft.

		Bei einem der Fälle ging es um einen Unfall, bei dem eine Flasche mit der
		Aufschrift \enquote{highly inflammable} neben einer Heizung abgestellt wurde
		der Arbeiter, dessen Muttersprache nicht Englisch war, ging davon aus, dass
		\enquote{inflammable} unbrennbar hieße, da \enquote{flammable} brennbar
		heißt. Da die Vorsilbe \enquote{in-} nicht immer das Gegenteil ausdrückt,
		kam es zum Entzünden der Flasche.\autocite{wiki:Benjamin_Lee_Whorf}

\chapter{Sapir-Whorf-Hypothese}
\label{chap:sawo_hypothese}
\section{Einleitung}
\label{sec:sawo_einleitung}
Die Sapir-Whorf-Hypothese ist eine, aus den Schriften Benjamin Lee Whorfs,
abgeleitete Annahme der Linguistik. Hierbei wird versucht zu ergründen, ob eine
Sprache, das heißt deren grammatikalische und lexikalische Strukturen, einen
Einfluss auf die Welterfahrung einer Sprachgemeinschaft hat. Die Nennung der
Sprachgemeinschaft ist in der Hinsicht wichtig, da die Welterfahrung im
Vergleich verschiedenener Personen immer in gewissem Maße abweicht. Wenn die
Welterfahrung jetzt allerdings bei einer Sprachgemeinschaft in eine ähnliche
Richtung ginge, so kann davon ausgegangen werden, dass die Sprache das Denken
beinflusst. So würde daraus folgen, dass es Gedanken einer Person gäbe, die nur
in ihrer Muttersprache verstanden werden können und für andere Sprachen ganzlich
unverständlich wären.

Auf diesem Konzept basieren die folgenden Interpretationen, die sich das Konzept
zum Axiom machen und somit von seiner Umabstreibarkeit ausgehen.
\autocite{wiki:Sapir-Whorf-Hypothese}
	\section{Whorf postum interpretiert}
	\label{sec:sawo_interpret}
	Wie bereits angesprochen, formuliert Whorf seine Konzepte nie zu der aktuellen
	bekannten Hypothese aus. Dies geschieht durch Ableitungen anderer Personen,
	die der dadurch entstandenen Theorie ihren Namen geben. Bei diesen
	Interpretationen entstehen zwei Ausrichtungen:

	\begin{enumerate}
		\item Das Prinzip der linguistischen Relativität
		\item Das Prinzip des linguistischen Determinismus
	\end{enumerate}
		\subsection{Das Prinzip der linguistischen Relativität}
		\label{sec:sawo_lingrelativ}
		Das Prinzip der linguistischen Relativität ist das gemäßigtere der beiden
		Prinzipien und die ursprüngliche, direkt auf Whorfs Schriften
		zurückzuführende Variante.
		\blockquote{
			\enquote{We dissect nature along lines laid down by our native languages. The
			categories and types that we isolate from the world of phenomena we do not
			find there because they stare every observer in the face; on the contrary,
			the world is presented in a kaleidoscopic flux of impressions which has to
			be organized by our minds – and this means largely by the linguistic systems
			in our minds. We cut nature up, organize it into concepts, and ascribe
			significances as we do, largely because we are parties to an agreement to
			organize it in this ways – an agreement that holds throughout our speech
			community and is codified in the patterns of language.}\autocite{article:Science_and_linguistics}
		}
		\blockquote{
			\enquote{Wir gliedern die Natur nach den Vorgaben unserer Muttersprachen. Die
			Kategorien und Typen, die wir aus der Welt der Phänomene isolieren, finden
			wir dort noch nicht vor, sie blicken jedem Betrachter als eigene Gegebenheit
			ins Gesicht. Die Welt stellt sich uns kaleidoskopartig als ein Fluss von
			Eindrücken dar, der von unserem Verstand erst organisiert werden muss – und
			das bedeutet weitgehend von den sprachlichen Strukturen unseres Verstandes.
			Wir zerschneiden die Natur, ordnen sie ein in Begriffe und weisen diesen
			Bedeutungen zu. Wir tun dies, als wären wir Teilnehmer einer Vereinbarung,
			alles erst auf diese Weise zu organisieren – einer Vereinbarung, die für
			unsere jeweilige Sprachgemeinschaft gilt und die bereits in unseren
			Sprachmustern kodifiziert ist.}\autocite{article:Science_and_linguistics}
		}
		Was Benjamin Whorf hier beschreibt, lässt sich im Grunde so zusammenfassen:
		\smallskip\newline
		Es gibt in der außersprachlichen Wirklichkeit unendlich viele Konzepte. All
		diese zu versprachlichen wäre ein Akt des Unmöglichen und so integriert eine
		Sprachgemeinschaft nur die Konzepte in ihre Sprache, die sie für wichtig
		hält.

		Dieses Filtern der Konzepte basiert sowohl auf den jeweiligen
		Umweltbedingungen, als auch auf den gesellschaftlichen Entwicklungen.
		Da sich diese beiden Faktoren in ständiger Veränderung befinden, kann eine
		gewisse Offenheit für neue Gedanken bzw. Konzepte und damit Wörter
		existieren.\autocite{wiki:Sapir-Whorf-Hypothese}

		\subsection{Das Prinzip des linguistischen Determinismus}
		\label{sec:sawo_lingdetermin}
		Eine radikalere Variante, oder besser, Interpretation der Sapir-Whorf-Hypothese
		ist das Prinzip des linguistischen Determinismus. Hierbei wird behauptet,
		die Sprache würde die Erkenntnis determieren. Es würde also eine
		Abhängigkeit der Weltansicht von der Sprache existieren. Diese Haltung ist
		im Wesentlichen eine Uminterpretation der Werke Whorfs:
		\blockquote{
			\enquote{language determining perception (cf. Sapir and Whorf)}\autocite[610--646]{article:LingSignific}\medskip\newline
			\enquote{In recent years the anthropologists Whorf […] have put forward the
			view that language is a determinant of perception and thought […] the
			semantic character of the form classes fixes the fundamental reality in a
			language community}\autocite[1--5]{article:LingDeterm}\medskip\newline
			\enquote{The structure of anyone’s native language strongly influences or
			fully determines the world-view he will acquire as he learns the language}\autocite[125--153]{article:Reference}
		}
		Das heißt also, dass eine Person nur das Denken könne, was auch durch die
		Sprache dieser Person ausgedrückt werden kann. Im Umkehrschluss hieße das,
		dass eine Person keine neuen Sachzusammenhänge begreifen könne, wenn die
		die sprachlichen Fähigkeiten das nicht zuließen.\autocite{article:Linguistic_Relativity}

		\subsection{Empirische Kritik}
		\label{sec:sawo_empkritik}
		Whorf gibt einige Belege aus eigener Forschung als Beweise seines Axioms an.
		Die meisten wurden durch spätere Untersuchungen wiederlegt.
		\begin{itemize}
			\item Die Sprache der Hopi (ein amerikanisches indigenes Volk) hätte nach
			Befragung eines Hopi-Sprecher aus New York keine
			Ausdrucksmöglichkeiten für zeitliche Einordungen. So würde dieses
			Volk das europäische Zeitkonzept von Vergangenheit, Gegenwart und
			Zukunft nicht verstehen.
				\begin{itemize}
					\item Ekkehart Molotki weist 1983 nach, dass das Hopi über ein komplexes
					Zeitformensystem verfügt.\autocite{wiki:Sapir-Whorf-Hypothese}
				\end{itemize}
				\item Die Sprache der [Inuit] hätte enorm viele Begriffe für Schnee.
				\begin{itemize}
					\item Das Inuit hat im Grunde nur zwei Formen für Schnee: \textit{aput}
					für fallenden Schnee und \textit{quana} für liegenden Schnee. Aus
					diesen beiden Stämmen lassen sich unermesslich viele neue Wörter
					bilden. Das funktioniert äquivalent mit allen anderen Wortstämmen.\autocite{wiki:Sapir-Whorf-Hypothese}
				\begin{figure}[!htb]
					\centering
					\includegraphics[width=0.33\textwidth]{inuitcartoon}
					\caption[Kritik an den Belegen Whorfs {\autocite{cartoon:Inuit}}]{Kritik an den Belegen Whorfs\footnotemark}
					\label{fig:sawo_inuitcartoon}
				\end{figure}
				\footcitetext{cartoon:Inuit}
				\end{itemize}
		\end{itemize}
		\subsection{Theoretische Kritik}
		\label{sec:sawo_theokritik}
		Das Prinzip des linguistischen Determinismus ist durch Gegenbeispiele leicht
		zu wiederlegen.
		\begin{itemize}
			\item Im Zusammenhang mit dem linguistischen Determinismus wird das
			Argument der \enquote{zeitenlosen} Hopi-Sprache genannt und behauptet,
			Sprechende dieser, könnten das Konzept von vorrüberstreichender Zeit nicht
			verstehen.
				\begin{itemize}
					\item Der völlig korrekte deutsche Satz \enquote{Kommst du morgen zum
					Kuchenessen vorbei?} macht selbst im Präsens deutlich, dass es sich um
					die Zukunft handelt.
				\end{itemize}
				\item Auch wenn es im Englischen keinen Begriff für \textit{Schadenfreude}
				gibt, können sich die meisten Englischsprachigen etwas unter dem Konzept,
				Freude an jemand anderes Schaden zu finden, vorstellen.\autocite{article:Linguistic_Relativity}
		\end{itemize}

	\section{Aktualität der Hypothese}
	\label{sec:sawo_aktualität}
	Der linguistische Determinismus wird seit spätestens 1980 als widerlegt
	angesehen.\autocite{article:Linguistic_Relativity2}
	Dennoch stehen auch heute noch versiedene Versionen der linguistischen
	Relativität im Raum, denn, dass etwas, so eng mit den Gedanken Verbundenes, wie
	Sprache keine Auswirkungen auf diese hätte, ist schwer vorstellbar.

	Aus diesem Grund wird weitere Forschung betrieben, die die alten Belege Whorfs
	ersetzen soll.
		\subsection{Neue empirische Forschung}
		\label{sec:sawo_empforschung}
		\subsubsection{Aborigine-Sprache: Guugu Yimithirr}
		\label{sec:sawo_guuguyim}
		Die Sprache der Aborigines aus dem Gebiet North Queensland gebraucht sich
		keiner egozentrischer Koordinaten. Die Begriffe \textit{links},
		\textit{rechts}, \textit{vor} oder \textit{hinter} gibt es nicht. Die
		Richtungsangaben basieren ausschließlich auf den Himmelsrichtungen. Das heißt,
		ein Gegenstand wäre nicht rechts von einer Person, sondern z. B. östlich, und das
		auch weiterhin, wenn sich die Person umdrehen würde. Das funktioniert auch
		beim Ansehen eines Films. Sollte der Fernseher nach Norden ausgerichtet
		sein und eine Person auf dem Bildschirm auf die Zuschauenden zukommen, würde
		die Person aus nördlicher Richtung kommen.

		Um sich der Himmelsrichtungen ständig bewusst zu sein, bedarf es einer ständigen
		Beobachtung der Umgebung, sowie einer genauen Erinnerung daran, wie man sich
		selbst bewegt hat. Mit einer Sprache die genaue Himmelsrichtungen abfragt,
		wird diese Art zu Denken von kindesbeinen auf trainiert, sodass Kinder dieses
		System mit 7 - 8 Jahren gemeistert haben.

		Diese Art, Richtungen anzugeben, ist keine Einzelerscheinung. Man findet sie
		unteranderem in Polynesien, Mexico, Namibia oder Bali.\autocite{article:Linguistic_Relativity}
		\newpage
		\subsubsection{Farbwahrnehmung}
		\label{sec:sawo_farbw}
		Auch die Wahrnehmung von Farben ist zum Teil durch die Sprache bestimmt.
		Wo es in Sprachen unterschiedliche Begriffe für die Farben Grün und Blau gibt,
		so gelten diese beiden Farben als Farbabstufungen in anderen.

		Ein Beispiel liefert hier eine Studie des \textit{MIT}, bei der die Tsimane, eine
		bolivische Urbevölkerung, an einem Experiment teilnamen, das klären sollte,
		wie Monolinguelle und Bilinguelle Farben einteilen.

		Hierbei wurden den Teilnehmenden im ersten Durchgang nacheinander 84
		verschieden farbige Chips gezeigt, die sie mit einem Farbbegriff beschreiben
		sollten.
		Im zweiten Durchgang wurden ihnen alle 84 Chips vorgelegt und sie sollten
		sie in Gruppen einteilen. Die Bilinguellen führten beide Tests auf Tsimane
		und auf Spanisch durch.

		Herausgestellt hat sich dabei, dass die Bilinguellen die Tsimane-Wörter
		\textit{yushñus} und \textit{shandyes} ausschließlich für Blau bzw. für Grün
		verwendeten, wärend die beiden Begriffe bei den Monolinguellen für Blau- und
		Grüntöne gleichermaßen verwendet werden.
		Das ist besonders interessant, da viele indigene Völker, und so auch die
		Tsimane, die wärmeren Farben stärker unterteilen als die Kälteren.
		Die Bilinguellen wurden außerdem präziser mit der Einordung von Gelb und Rot,
		als auch Braun.

		Im spanischen Test benutzten die Bilinguellen übrigens die standardmäßigen
		spanischen Farbbegriffe.\autocite{article:Color_Perception}

\chapter{Fazit}
\label{chap:fazit}
Benjamin Lee Whorf hat mit seinen Schriften Wellen in die Linguistik geschlagen,
und ist auch heute noch Anlass für so manche Diskussion. Die Prinzipien, die
auf ihm und seinem Lehrer Edward Sapir aufbauen, sind wenigstens zur Hälfte,
in Form der linguistischen Relativität, heute noch in Gebrauch und werden mit
neuen Studien überprüft.

Denn man muss zugeben, dass die Studien Whorfs
handwerklich schlecht gemacht sind. Das \enquote{100-Wörter-für-Schnee-Argument}
basiert auf einem Übersetzungsfehler und das über das Hopi-Zeitensystem auf einem
einzigen Hopibewanderten, der kein Muttersprachler war. Das lindert auch der Fakt,
dass die Studien in den frühen 1900er Jahren durchgeführt wurden, nur wenig.

Neue Forschung dagegen ist qualitativ besser und bedient sich auch anderen
Techniken. Wenn Sprache also die Gedanken beeinflusst, müssen Studien ohne den
Input oder Output von Sprachen zum Ergebnis kommen. Das heißt es müssen Spiele,
wie in \texttt{\ref{sec:sawo_farbw} Farbwahrnehmung} zum Einsatz kommen.

\printbibliography

\end{document}
